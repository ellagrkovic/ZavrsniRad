\documentclass{fer}
\usepackage[utf8]{inputenc}
\usepackage[T1]{fontenc}
\usepackage[croatian]{babel}
\usepackage{graphicx}
\usepackage{amsmath}
\usepackage{booktabs}

\title{Algoritam za raspoređivanje zadataka za više robota uz pokretnu traku}
\author{Ella Grković}
\date{Zagreb, srpanj 2025.}

\begin{document}

\maketitle

\chapter{Uvod}

Razvoj robotike posljednjih desetljeća ima snažan utjecaj na industriju, gospodarstvo i svakodnevni život. Roboti se danas koriste u raznim granama – od automatizirane proizvodnje i logistike, do medicine, istraživanja svemira i kućne upotrebe. U industrijskom okruženju, posebno u kontekstu pametnih tvornica i automatizirane proizvodnje, primjena robota postaje sve sofisticiranija i sveprisutan je zahtjev za njihovim učinkovitim upravljanjem i suradnjom.

Industrijski roboti već odavno ne obavljaju samo jednostavne, ponavljajuće zadatke. Umjesto toga, sve se više integriraju u kompleksne sustave gdje je važno planiranje, koordinacija i prilagodba zadacima u stvarnom vremenu. U takvim sustavima, u kojima više robota djeluje istovremeno, ključno je osigurati optimalnu raspodjelu zadataka kako bi se postigla maksimalna učinkovitost, pouzdanost i fleksibilnost.

\chapter{Opis problema}

Jedan od konkretnih i čestih scenarija u industrijskoj automatizaciji uključuje robotske ruke koje preuzimaju predmete s pokretne trake. Takvi sustavi nalaze primjenu u sortirnim centrima, pakirnicama, prehrambenoj industriji i drugim okruženjima u kojima je važno brzo i precizno manipulirati objektima u pokretu.

Upravo takav problem predstavlja temelj ovog završnog rada, koji se fokusira na razvoj algoritma za raspoređivanje zadataka više robota uz pokretnu traku. Ključni izazov u ovakvim sustavima jest kako učinkovito dodijeliti zadatke – odnosno predmete na traci – raspoloživim robotskim rukama. Budući da se objekti ne pojavljuju istovremeno, a njihova pozicija varira, potrebno je u stvarnom vremenu donositi odluke koji će robot pokupiti koji predmet. Ako se zadaci ne dodijele optimalno, može doći do preklapanja, neiskorištenosti robota ili propuštanja objekata, što smanjuje ukupnu učinkovitost sustava.

Problem se stoga može promatrati kao optimizacijski problem dodjele zadataka u kojem je cilj maksimizirati određene pokazatelje performansi, kao što su broj uspješno pokupljenih predmeta, iskorištenost robotskih ruku i uravnoteženost opterećenja među robotima. Rješenje ovog problema nije trivijalno, jer uključuje više čimbenika poput položaja objekata, vremena dolaska, dosega robota i ograničenja u njihovom djelovanju.

U ovom završnom radu pristupa se ovom izazovu kroz izgradnju i analizu algoritama za dodjelu zadataka više robota uz pokretnu traku. Bit će implementirana dva različita pristupa – jedan temeljen na heuristikama (npr. pravila prioriteta), a drugi na formalnijim metodama optimizacije (npr. dinamičko programiranje). Njihova učinkovitost analizirat će se na temelju simuliranih ulaznih podataka, s ciljem boljeg razumijevanja kompromisa između preciznosti, brzine izvođenja i složenosti algoritama.

\chapter{Rezultati}

U sklopu ovog rada implementirana su dva pristupa za dodjelu zadataka robotima koji preuzimaju pakete s pokretne trake: pohlepan algoritam, koji u svakom trenutku donosi lokalno optimalnu odluku, te optimalni algoritam, koji koristi optimizacijski pristup s ciljem maksimizacije ukupnog broja preuzetih paketa.

Evaluacija algoritama provedena je na unaprijed generiranim skupovima podataka koji sadrže 5, 10 i 15 paketa po testnom skupu. Rezultati su prikazani boxplot dijagramima i sažeti u tablicama.

\begin{table}[h]
\centering
\caption{Usporedba rezultata – pohlepni vs. optimalni}
\begin{tabular}{@{}llllll@{}}
\toprule
Broj paketa & Algoritam & Prosjek & Medijan & Min & Max \\
\midrule
5 & Pohlepan & 4.5 & 5 & 3 & 5 \\
5 & Optimalni & 5.0 & 5 & 5 & 5 \\
10 & Pohlepan & 7.2 & 7 & 4 & 9 \\
10 & Optimalni & 9.5 & 10 & 8 & 10 \\
15 & Pohlepan & 9.1 & 9 & 7 & 11 \\
15 & Optimalni & 12.6 & 13 & 10 & 14 \\
\bottomrule
\end{tabular}
\end{table}

Za dodatnu evaluaciju optimalnog algoritma testiran je njegov učinak kod različitih početnih rasporeda robota (standardni i obrnuti). Rezultati pokazuju da položaj robota može utjecati na broj uspješno preuzetih paketa, osobito pri većem broju paketa.

\begin{table}[h]
\centering
\caption{Učinak rasporeda robota}
\begin{tabular}{@{}llllll@{}}
\toprule
Broj paketa & Konfiguracija & Prosjek & Medijan & Min & Max \\
\midrule
5 & Optimalni & 5.0 & 5 & 5 & 5 \\
5 & Obrnuti & 5.0 & 5 & 5 & 5 \\
10 & Optimalni & 9.5 & 10 & 8 & 10 \\
10 & Obrnuti & 10.0 & 10 & 10 & 10 \\
15 & Optimalni & 12.6 & 13 & 10 & 14 \\
15 & Obrnuti & 14.1 & 14 & 13 & 15 \\
\bottomrule
\end{tabular}
\end{table}

\end{document}
